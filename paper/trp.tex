\documentclass{article}
\usepackage{float}

\begin{document}
\title{Transactional Reactive Programming}
\author{Brian Hurt \\
CTO and Chief Architect \\
Chatwisely}

\maketitle

\begin{quotation}
\textit{Sometimes, the elegant implementation is a function. Not a method. Not a class. Not a framework. Just a function.}
\begin{flushright}
-- John Carmack
\end{flushright}
\end{quotation}

\begin{abstract}
Transactional Reactive Programming (TRP) is a new approach to dynamic
programming and state management in functional programs.  Instead of
modeling changes in the state of the program as functions on a
continuous time, it instead models changes in state as atomic
transactions.  This allows for a more efficient implementation and a
better model of how state actually changes.
\end{abstract}

\section{Functional Reactive Programming}

Functional Reactive Programming (FRP) has gained some traction in the
functional programming community for the development of user interfaces.
A true introduction to Functional Reactive Programming is beyond the
scope of this paper.  Instead, we present a sketch of the idea in the
broadest strokes, mainly to lay down the terminology we use for the rest
of the paper.

The core idea of FRP is that we are modelling a system over continuous
time.  The core type of FRP is then a {\texttt Behavior a}, which has a
value of type \verb|a| at any given time.  It can be considered a
special type of function of type \verb|time -> a|.

For example, a specific behavior might hold the height of a given
window.  If we want the window to "pop open" at time t, the behavior's
value might look like:

\begin{center}
\setlength{\unitlength}{0.14in}
\begin{picture}(26,8)
\put(0,5){Height}
\put(11.8,0.8){Time}
\put(5,3){\line(0,1){5}}        % left border
\put(5,3){\line(1,0){20}}       % bottom border
\put(5,3.5){\line(-1,0){0.4}}   % tick line for the 0 label
\put(4,3.2){0}
\put(12,3){\line(0,-1){0.3}}    % tick line for the t_0 label
\put(11.8,2){$t_0$}
\put(15,3){\line(0,-1){0.3}}    % tick line for the t_1 label
\put(14.8,2){$t_1$}
\put(5,3.5){\line(1,0){7}}      % 0 line
\put(12,3.5){\line(1,1){3}}     % slope line
\put(15,6.5){\line(1,0){10}}    % open line
\put(5,6.5){\line(-1,0){0.4}}
\put(3.2,6.2){\small full}
\end{picture}
\end{center}

The window has height 0 from the begining of time to time $t_0$.  At that
point, it's height increases at a constant rate from 0 until it reaches it's
full height at time $t_1$.  At that point, it remains that height until the
end of time.

To deal with discrete events, such as mouse clicks or keystrokes, FRP
introduces the idea of Events.  And event can be considered a special
type of Behavior, where an {\texttt Event a} maps to a 
{\texttt Behavior (Maybe a)}, where the behavior is {\texttt Just a} if
the event is firing at that moment in time, or {\texttt Nothing} if it
is not.

In addition, most FRP libraries also provide a {\texttt Dynamic} type,
which is a combination of a Behavior and an Event, with the additional
requirement that the event fires when the behavior changes value.

TODO: write about the DOM monads here.

\section{The Unfullfilled Promise of FRP}

FRP as a concept can be applied to a large number of different domains.  In
practice, however, it is mainly used for web applications.

A common problem all WebUI frameworks in all languages have to deal with
is that modifying the DOM is expensive.  The standard solution is to
use a virtual DOM.  The virtual DOM is made up of plain old Javascript
objects, and is thus fast and cheap to create.  The framework user
supplies a function with a type \verb|ProgramState -> VirtualDOM|, which
is called whenever the program state is changed.  The newly created
virtual DOM is the compared to the current DOM, and whatever changes
needed are made. keeping the changes as minimal as possible.  It is then
just a ``simple matter of programming''\footnote{If this phrase does not
fill you with dread, you haven't been in the industry long} to optimize
the process.  You see this pattern with javascript frameworks like react
and angular, with Haskell frameworks like Shapdoinkle, and even with
non-web frameworks like Flutter.

The problem with this approach is that creating the virtual DOM and
comparing it to the real DOM is always going to be a lot of work.  Most
of which is unnecessary- especially if all you are doing is changing
an icon from \verb|bell-outline| to \verb|bell-filled| and adding a
\verb|style="color: red;"| attribute.

The promise of FRP is that we can skip the whole virtual DOM.  Instead,
the program state is distributed out to the widgets via events and
behaviors, like water through irrigation canals.  Even with the inherent
performance disdvatange that comes with cross compiling from a radically
different language FRP should have a significant performance advantage.

And yet, FRP does not fulfill this promise.  FRP has earned a reputation
for being slow.  It is past time to ask why.

\subsection{The Continuous Nature}

For all that behaviors are the central type for FRP, they are noticeably
rare in web applications.  Events and dynamics are everywhere, but behaviors
are rare.

This is because the continuous nature of behaviors is not necessary for web
applications.  For example, in defining what a behavior was, we used the
example of opening a window:

\begin{center}
\setlength{\unitlength}{0.14in}
\begin{picture}(26,8)
\put(0,5){Height}
\put(11.8,0.8){Time}
\put(5,3){\line(0,1){5}}        % left border
\put(5,3){\line(1,0){20}}       % bottom border
\put(5,3.5){\line(-1,0){0.4}}   % tick line for the 0 label
\put(4,3.2){0}
\put(12,3){\line(0,-1){0.3}}    % tick line for the t_0 label
\put(11.8,2){$t_0$}
\put(15,3){\line(0,-1){0.3}}    % tick line for the t_1 label
\put(14.8,2){$t_1$}
\put(5,3.5){\line(1,0){7}}      % 0 line
\put(12,3.5){\line(1,1){3}}     % slope line
\put(15,6.5){\line(1,0){10}}    % open line
\put(5,6.5){\line(-1,0){0.4}}
\put(3.2,6.2){\small full}
\end{picture}
\end{center}

This is not how windows pop open in a web application.  Instead of directly
controlling the window height, we would use a CSS animation.  At time $t_0$,
we could simply change the class of the element from ``closed'' to ``open''.
The graph of our behavior's value would now look like this:

\begin{center}
\setlength{\unitlength}{0.14in}
\begin{picture}(26,8)
\put(0,5){Class}
\put(11.8,0.8){Time}
\put(5,3){\line(0,1){5}}        % left border
\put(5,3){\line(1,0){20}}       % bottom border
\put(5,3.5){\line(-1,0){0.4}}   % tick line for the closed label
\put(1.2,3.3){\small ``closed''}
\put(5,6.5){\line(-1,0){0.4}}   % tick line for the open labl
\put(1.6,6.3){\small ``open''}
\put(12,3){\line(0,-1){0.3}}    % tick line for the t_0 label
\put(11.8,2){$t_0$}
\put(5,3.5){\line(1,0){6.8}}    % closed line
\put(12,6.5){\line(1,0){10}}    % open line
\put(12,3.5){\circle{0.4}}      % closed circle
\put(12,6.5){\circle*{0.4}}     % open circle
\end{picture}
\end{center}

Note that the change of the class of the HTML element is discrete.  There
are no intermediate states- one instant, the class is ``closed'', the next
``open''.

Now, it's not hard to model discrete events in a continuous system.  But the
continuous nature of FRP imposes costs on the implementation.  For example,
a given event can not fire multiple times at one point in time.  Enforcing
this constraint causes complexity and reduces performance.  As we shall see,
relaxing this constraint allows for both a simpler and more performant
implementation.

\subsection{The Direction of Events}

Let's consider the following situation: we start with an \verb|Event a|.
We then \verb|fmap| it to create an \verb|Event b|.  We now have two
separate objects in memory, \verb|Event a| and \verb|Event b|, and can
ask what references these objects have.

Now, obviously, \verb|Event b| needs to contain a reference to
\verb|Event a|.  We do not want the garbage collector to free
\verb|Event a| before \verb|Event b| is freed.  At the same time, the
information (values) flow from \verb|Event a| to \verb|Event b|- when
\verb|Event a| fires, the system needs to apply the function passed to
\verb|fmap| and fire \verb|Event b|.  So \verb|Event a| needs a
reference to \verb|Event b|.  Pictorially, we have:

\begin{figure}[H]
\setlength{\unitlength}{0.14in}
\centering
\begin{picture}(17.2,3)
% \put(0,0){\line(0,1){3}}
% \put(0,0){\line(1,0){17.2}}
% \put(17.2,3){\line(0,-1){3}}
% \put(17.2,3){\line(-1,0){17.2}}

\put(1,1.2){\texttt Event a}
\put(2.6,1.5){\oval(5,2)}
\put(13,1.2){\texttt Event b}
\put(14.6,1.5){\oval(5,2)}

\put(7,2.1){\small ``owns''}
\put(6.1,0.2){\small sends events}

\thicklines
\put(12.2,1.8){\vector(-1,0){7.1}}
\put(5,1.0){\vector(1,0){7.1}}
\end{picture}
\end{figure}

Now, a naive implementation that used normal ("strong") references would
end up not allowing either event to be collected until all events are
collected.  This is a recipe for galloping memory leaks and degrading
performance.  The normal solution to this is to make one of the
references (generally, the reference \verb|Event a| holds to
\verb|Event b|) a weak reference that does not prevent garbage collection.

There is a problem with using weak references in events.  The runtime
implementors assume, generally correctly, that weak references will not
be common, and not be critical for performance.  Indeed, javascript (a
very important target for most FRP systems) doesn't directly support
weak references at all, requiring other implementations with even
greater complexity and cost.  Given the ubuiquity of events in FRP
programming, this rapidly becomes a critical problem.

\section{Introducing Triggers}

The problem is that reference that \verb|Event a| needs to hold to
\verb|Event b|.  If only we could reverse the flow of information, so
that our picture looked like this:


\begin{figure}[H]
\setlength{\unitlength}{0.14in}
\centering
\begin{picture}(17.2,3)
% \put(0,0){\line(0,1){3}}
% \put(0,0){\line(1,0){17.2}}
% \put(17.2,3){\line(0,-1){3}}
% \put(17.2,3){\line(-1,0){17.2}}

\put(1,1.2){\texttt Event a}
\put(2.6,1.5){\oval(5,2)}
\put(13,1.2){\texttt Event b}
\put(14.6,1.5){\oval(5,2)}

\put(7,2.1){\small ``owns''}
\put(6.1,0.2){\small sends events}

\thicklines
\put(12.2,1.8){\vector(-1,0){7.1}}
\put(12.2,1.0){\vector(-1,0){7.1}}
\end{picture}
\end{figure}


Then we wouldn't need weak references at all.  \verb|Event b| would
simply hold a normal ("strong") reference to \verb|Event a|, and 
\verb|Event a| wouldn't hold a reference of any sort to \verb|Event b|.

But then these are no longer events we are talking.  Instead of
producing values when things happen, they consume values.  They are
co-events, the contravariant version of events.  We call co-events
triggers.
 
The key function of Triggers is \verb|fire|:

\begin{verbatim}
    fire :: Trigger m a -> a -> m ()
\end{verbatim}

The \verb|fire| function fires the trigger with a given value.  The
monad parameter is the moment monad the trigger fires in.  This allows
us to provide multiple different implementations for differing
situations.

\section{Working with Triggers}

Switching to Triggers from Events is nearly mechanical:

\begin{itemize}
\item If a function takes an Event as an input, it now returns a
Trigger.
\item If a function returns an Event, it now takes a Trigger.
\end{itemize}

Some examples of how this works.

\subsection{Example: hold}

Consider the function \verb|hold|\footnote{Simplified to keep irrelevant
details from obscuring the point}:

\begin{verbatim}
    hold :: a -> Event m a -> IO (Behavior m a)
\end{verbatim}

This function creates a behavior given an initial value and an event.
Every time the event fires, the behavior is given a new value. 

We apply the mechanical transform, changing the event to a trigger, and
changing it from being an input to being an output:

\begin{verbatim}
    hold :: a -> IO (Behavior m a, Trigger m a)
\end{verbatim}

Now our function takes an initial value for the behavior to hold, and
returns both the behavior and a trigger that, when fired, updates the
behavior.

\subsection{Example: split}

The split function converts an event that produces an \verb|Either|
value, into a pair of events:

\begin{verbatim}
    split :: Event (Either a b) -> (Event a, Event b)
\end{verbatim}

Applying our transform, the two returned events become input triggers,
and the input event becomes the output trigger\footnote{I don't perform
the natural currying here, to emphasize how values move}:

\begin{verbatim}
    split :: (Trigger m a, Trigger m b) -> Trigger m (Either a b)
\end{verbatim}

Remember that triggers consume values.  So when the result trigger is
fired, it's given an \verb|Either|, which it uses to select which of the
two input triggers is then fired.

But this function already exists in the Haskell ecosystem: it is simply
a variant on the \verb|choose| function from the contravariant library:

\begin{verbatim}
    choose :: Decidable f
                => (a -> Either b c) -> f b -> f c -> f a
\end{verbatim}

Triggers are \verb|Decidable| (as well as \verb|Divisible| obviously).

\section{A Change in the Model}

To implement Triggers, we consider the simplest possible implementation,
which is just a newtype:

\begin{verbatim}
    newtype Trigger m a = Trigger { fire :: a -> m () }
\end{verbatim}

A similar argument can be made with Behaviors:
\begin{verbatim}
    newtype Behavior m a = Behavior { sample :: m a }
\end{verbatim}

The \verb|hold| function becomes just a wrapper around a mutable
variable.  For example, if use \verb|STM| for our moment monad, then
\verb|hold| becomes:

\begin{verbatim}
    hold :: a -> IO (Behavior STM a, Trigger STM a)
    hold start = do
        tvar <- newTVarIO start
        return (Behavior (readTVar tvar),
                    Trigger (writeTVar tvar))
\end{verbatim}


This is an ernormous simplification over the implementations of
classical FRP events.  It comes, however, with a change in the mental
model of FRP.

In ``classical'' FRP, moments represent a single, unique, point in time.
This means that, in any given moment, a behavior can only have a single
value, and that an event can only fire once.

This simplified model violates it.  Events can fire multiple times in a
single moment, and events can take multiple different values.  The
easiest demonstration of this is:

\begin{verbatim}
    problematic :: IO (Behavior (Int, Int, Int))
    problematic = do
        (beh, trig) <- hold 0
        return Behavior $ do
            i1 <- sample beh
            fire trig (i1 + 1)
            i2 <- sample beh
            fire trig (i2 + 1)
            i3 <- sample beh
            fire trig (i3 + 1)
            return (i1, i2, i3)
\end{verbatim}

Here, we are firing the trigger that sets the behavior multiple times in
a single moment.  Furthermore, we are sampling the behavior multiple
times, and getting three different results.

Despite the fact that this is not the standard model for FRP doesn't
mean that it's not a valid, even familiar model.  Instead of a moment
being a specific moment in time, a moment is instead a transaction.


\section{Extra Text}

There is a temptation to define \verb|runTrigger| to have the more
general type \verb|m a -> IO a|.  But this eliminates an interesting
optimization opporutinity.

One common situation with FRP is wanting to run the same code on both
the frontend and the backend.  This is so the backend can prerender
the HTML.  In this case, we can use the fact that \verb|Proxy| is monad
to our advantage.  

The fact that \verb|Proxy| is a monad is one of those things that, when
you first notice them, you tend to dismiss as technically correct but
essentially silly.  But once you realize how and why they are useful
they suddenly become brilliant.

We can define implementations of all of our type classes for Proxy:

\begin{verbatim}

    instance MonadHold STM where
        hold _init = do
            pure (Behavior Proxy, Trigger (const Proxy))

    instance RunTrigger Proxy where
        runTrigger Proxy = pure ()

\end{verbatim}

These ``trivial'' implementations just side step doing everything.  But
this is exactly the behavior we want when we're doing a prerender.

We can even define a type class to let us detect that we are doing a
prerender, to have some DOM only generated in a prerender (such as a
"please wait while we're loading" popup, or spinners), or only generated
when not in a prerender:

\begin{verbatim}

    class IsPrerender m where
        isPrerender :: Proxy m -> Bool

    instance IsPrerender Proxy where
        isPrerender Proxy = True

    instance IsPrerender IO where
        isPrerender Proxy = False

    instance IsPrerender STM where
        isPrerender Proxy = False

    instance (IsPrerender m, MonadTrans t)
        => IsPrerender (t m) where
            isPrerender Proxy =
                isPrerender (Proxy :: Proxy m)

\end{verbatim}

\section{Dynamics}





\end{document}

